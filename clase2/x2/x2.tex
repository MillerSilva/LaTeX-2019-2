\documentclass{report}
\usepackage[utf8]{inputenc}

\usepackage{graphicx}
\usepackage{here}

\newcommand*{\fd}{\longrightarrow}

\newcommand{\sbf}{\sffamily\bfseries}

\newcommand{\impo}[1]{\textbf{\underline{#1}}}

\newcommand{\parc}[2]{\frac{\partial#1}{\partial#2}}

\newcommand{\pl}{\ensuremath{x_1}}

\begin{document}
	
texto \pl\ texto2 $\pl$ texto3

$$
a\fd b
$$

$$
\parc{f}{x}=\frac{\partial f}{\partial x}
$$

texto \impo{importante} texto

{\sbf hola mundo} texto
	
\setcounter{chapter}{4}
	
\chapter{Mi primera matemática}
	
texto texto textoa $a+b+c$ texto texto texto texto texto texto texto texto texto texto texto texto texto texto $8-5$ texto texto \(3+4\) textotexto texto textotexto texto texto	 
\[a+b-c(d)+1\]
texto texto textoa $a+b+c$ texto texto texto texto texto texto texto texto texto 
$$
6+x+y+7
$$
texto texto textoa $a+b+c$ texto texto texto texto texto texto texto texto texto 
\begin{equation}
a+b+c=5>4,\quad 3<6
\end{equation}

$$
a_1+ a_{22}+a^4+ a^{34}
$$

$$
\frac{34}{a+b}
$$

$$
\sqrt{x}\qquad \sqrt[n]{y}
$$

$$
\sum_{i=1}^n a_i
$$

$$
\int_a^b f(x)dx
$$

$$
\int\limits_a^b f(x)dx
$$

$$
\sum\nolimits_{i=1}^n a_i
$$
	
texto... \qquad texto\dots 
	
$$
a,b,\ldots,c,d\in A
$$

$$
a+b+c+\cdots+z =90
$$

$$
\vdots\qquad \ddots
$$

$$
\alpha +\beta + \gamma=\theta \neq \vartheta\in \Gamma
$$

$$
5\le 6\leq 7, 7\geq 0
$$

$$
f(5)\to f(8)\longrightarrow f(9)
$$

\newpage

texto texto $\int_a^b f(x)dx\bigcup_{i=1}^n$ texto
$$
\int\bigcup_{i=1}^n
$$

$$
\cos\alpha\neq cos\alpha
$$

$$
\lim_{x\to 0}\frac{1}{x}=\infty
$$

$$
\tilde{x}\quad \widetilde{abc}
$$

$$
\hat{a}\qquad \widehat{abc}
$$

$$
\dot{a}\quad\vec{b}\quad\bar{c}
$$

$$
\left(\frac{a}{b}\right)
$$

$$
\left(\frac{a}{b}\right.
$$

$$
f(x)=g(x) \mbox{ cuando } x\in X
$$

$$
xyz\quad \mathrm{xyz}\quad \mathsf{xyz}\quad \mathtt{xyz}\quad\mathit{xyz}\quad\mathbf{xyz} \quad\mathcal{XYZ}
$$

$$
\left(\begin{array}{ccc}
a & b & c \\
d & e & f \\
g & h & i
\end{array}\right]
$$

$$
f(x)=\left\{\begin{array}{ll}
x^3 , & x \in A \\
x^2, & x\in B
\end{array}\right.
$$

$$
\underline{abdc}\quad\overline{xyz}
$$

$$
\underbrace{3+3+3+3}_{=12}\quad \overbrace{4+4+4+4}^{=16}
$$

$$
5+6\stackrel{\mathrm{def}}{=}11
$$

\addtocounter{equation}{20}

\begin{eqnarray}
a+b+c&=&d+e \\
5&=&6 \\
p+m&=&3
\end{eqnarray}

\begin{eqnarray*}
a+b+c&=&d+e \\
5&=&6 \\
p+m&=&3
\end{eqnarray*}

$$
f(x)=g(x), \;\forall x\in X
$$

$$
a\;b\:c\,d\!e
$$

$$
x^{x^x} \scriptstyle{x}\quad\scriptscriptstyle{x}
$$

texto texto texto texto $\displaystyle\sum_{i=1}^n a_i$ texto texto texto texto texto texto texto texto texto texto texto texto texto texto texto texto texto texto texto texto texto texto texto texto texto texto texto texto texto texto

$$
\int\quad \textstyle{\int_a^b f(x)dx}
$$

$$
\big(\frac{a}{b}\big)\quad \Big(\frac{a}{b}\Big)\quad \bigg(\frac{a}{b}\bigg)\quad \Bigg(\frac{a}{\sum_a^b b}\Bigg)
$$



	
	
	
	
	
	
	
\addtocounter{chapter}{4}	
	
	
\chapter{Terminando letras}
\setcounter{page}{90}
	
	\listoftables
	
	\ \\[2cm]

Seguramente lo habrás aprendido en la escuela primaria: la Tierra describe una órbita elíptica alrededor del Sol.

Este recorrido, que se conoce como movimiento de traslación, le toma al planeta unos 365 días 
(más 5 horas, 45 minutos y 46 segundos).

El otro movimiento que te enseñaron es el de rotación: la Tierra gira en torno a su propio eje.

\begin{figure}[H]
\centering
\includegraphics[width=5cm]{uni}
\caption{EL escudo oficial de la UNI}\label{fig1}
\end{figure}

En la Figura \ref{fig1} tenemos el escudo de la UNI...

\begin{enumerate}
\item naranjas
\item peras
\begin{enumerate}
	\item naranjas
	\item peras
	\item manzanas
	\begin{enumerate}
		\item naranjas
		\item peras
		\item manzanas
		\item nísperos
		\begin{enumerate}
			\item naranjas
			\item peras
			\item manzanas
			\item nísperos
		\end{enumerate}
	\end{enumerate}
	\item nísperos
\end{enumerate}
\item manzanas
\item nísperos
\end{enumerate}

\begin{itemize}
	\item naranjas
\item peras
\item manzanas
\begin{itemize}
	\item naranjas
	\item peras
	\item manzanas
	\item nísperos\footnote{fruta que no conozco}
\end{itemize}
\item nísperos
\end{itemize}

\begin{description}
\item[PALABRA1] Seguramente lo habrás aprendido en la escuela primaria: la Tierra describe una órbita elíptica alrededor del Sol.

\item[TEXTO2] Seguramente lo habrás aprendido en la escuela primaria: la Tierra describe\footnote{esto es un pie de página}
\end{description}

\begin{tabular}{rcll}
naranja & pera & manzana & níspero \\
naranja2 & pera2 & manzana2 & níspero2 \\
naranja3 & pera3 & manzana3 & níspero3 \\
\end{tabular}

\ \\[1cm]

\begin{tabular}{|r|c|l|l|}
	naranja & pera & manzana & níspero \\
	naranja2 & pera2 & manzana2 & níspero2 \\
	naranja3 & pera3 & manzana3 & níspero3 \\
\end{tabular}

\ \\[1cm]

\begin{tabular}{rcll}
	\hline
	naranja & pera & manzana & níspero \\
	\hline
	naranja2 & pera2 & manzana2 & níspero2 \\
	\hline
	naranja3 & pera3 & manzana3 & níspero3 \\
	\hline
\end{tabular}

\ \\[1cm]

\begin{table}[H]
	\centering
\caption{Esta es mi primera tabla flotante}
\begin{tabular}{|r|c|l|p{3cm}|}
	\hline
	naranja & \multicolumn{2}{c|}{PALABRA} & níspero \\
	\hline
	naranja2 & pera2 & manzana2 & níspero2 \\
	\cline{3-4}
	naranja3 & pera3 & manzana3 & níspero3 \\
	\hline
\end{tabular}
\end{table}

\begin{table}[H]
	\centering
	\caption{Este es mi flotante 2}\label{tab1}
	\begin{tabular}{|r|c|l|p{3cm}|}
		\hline
		naranja & \multicolumn{2}{c|}{PALABRA} & níspero \\
		\hline
	\end{tabular}
\end{table}

En la Tabla \ref{tab1} tenemos... 

Pero aprendí integración en \cite{ven14} y del sol en \cite[cap. 5]{smi18}

\begin{thebibliography}{99}
\bibitem{ven14} Venero, Armando. \emph{Análisis Matemático 2}. Gemar, 2014.

\bibitem{smi18} Smith, John. \emph{Acerca del sol}. Juntos, 2018.
\end{thebibliography}

\rule{5cm}{1mm}

\rule{2cm}{2cm}

\fbox{esto está encerrado}

\setlength{\fboxrule}{1mm}
\fbox{esto está encerrado}

\fbox{texto}

\setlength{\fboxsep}{6mm}

\fbox{esto está encerrado}

\setlength{\fboxrule}{0.4pt}
\setlength{\fboxsep}{3pt}
\fbox{esto está encerrado}

\fbox{\parbox{7cm}{Seguramente lo habrás aprendido en la escuela primaria: la Tierra describe una órbita elíptica alrededor del Sol. Seguramente lo habrás aprendido en la escuela primaria: la Tierra describe una órbita elíptica alrededor del Sol. Seguramente lo habrás aprendido en la escuela primaria: la Tierra describe una órbita elíptica alrededor del Sol.}}

\ \\[1cm]

\begin{minipage}{7cm}
Seguramente lo habrás aprendido en la escuela primaria: la Tierra describe una órbita elíptica alrededor del Sol. Seguramente lo habrás aprendido en la escuela primaria: la Tierra describe una órbita elíptica alrededor del Sol. Seguramente lo habrás aprendido en la escuela primaria: la Tierra describe una órbita elíptica alrededor del Sol.	
\end{minipage}

\ \\[1cm]

texto1 \raisebox{-2mm}{texto2} \raisebox{2mm}{texto3} texto4

\vfill

hola, estoy en la página \arabic{page} del cap. \Roman{chapter}, \thepage \hfill mundo

nombre:\hrulefill

apellido:\dotfill

\newpage

\begin{titlepage}
\centering\bfseries
{\Large UNIVERSIDAD NACIONAL DE INGENIERÍA\\[5mm]

FACULTAD DE CIENCIAS\\[5mm]

ESCUELA PROFESIONAL DE CIENCIAS}\\[8mm]

\includegraphics[width=5cm]{uni}\\[8mm]

{\large Trabajo presentado por:}\\[5mm]

{\Huge Juan Pérez}\\[8mm]

{\large Asesor:}\\[5mm]

{\Huge  John Smith}\\[8mm]

{\LARGE UNI--Lima\\[5mm]

2019}
\end{titlepage}

\end{document}