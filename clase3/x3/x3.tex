\documentclass{report}
\usepackage{color}

\newenvironment{impo}{\begin{center}		\sffamily\bfseries\color{blue}}{\end{center}}



%\usepackage[intlimits,leqno]{amsmath}
\usepackage{amsmath}
\numberwithin{equation}{section}

\DeclareMathOperator{\sen}{sen}

\DeclareMathOperator*{\algo}{algo}


\usepackage{amssymb}

\usepackage{showframe}

\usepackage{amsthm}

\newtheorem{eje}{Ejemplo}
\theoremstyle{definition}
\newtheorem*{cor}{Corolario}
\newtheorem{teo}{Teorema}[section]
\swapnumbers
\newtheorem{prop}{Proposición}[chapter]
\theoremstyle{remark}
\newtheorem{enun}[prop]{Enunciado}

\renewcommand{\proofname}{\upshape{\bfseries Prueba}}

\renewcommand{\qedsymbol}{$\heartsuit$}

\begin{document}
\chapter{teoremas}
\section{parte uno}
	
\begin{cor}
	esto es un corolario
\end{cor}	

\begin{teo}
	esto es un teorema
\end{teo}
\begin{prop}
esto es una proposición	
\end{prop}
\begin{enun}
enunciando algo	
\end{enun}
\begin{prop}
	esto es una proposición	
\end{prop}	

\begin{proof}[Demostración]
	esto es una prueba
\end{proof}

\begin{proof}
	otra prueba
\end{proof}
	
	
$$
\mathbb{ABCDEFGHIJKLMNOPQRSTUVWXYZ}
$$	

$$
\mathfrak{ABCDEFGHIJKLMNOPQRSTUVWXYZ}
$$

$$
\bigcup_{i\in I}A_i, A_i\in\mathfrak{D}
$$
	
	
	
	
	
	
	
	
	
	
	
	
texto texto texto $(\begin{smallmatrix} 4 & 5 \\ 7 & 8 \end{smallmatrix})$ texto texto texto texto 	
$$
\begin{matrix}
a & b & cv \\
5 & 4 & 9 \\
f & w & w
\end{matrix}
$$

$$
\begin{pmatrix}
a & b & cv \\
5 & 4 & 9 \\
f & w & w
\end{pmatrix}
$$

$$
\begin{bmatrix}
a & b & cv \\
5 & 4 & 9 \\
f & w & w
\end{bmatrix}
$$

$$
\begin{Bmatrix}
a & b & cv \\
5 & 4 & 9 \\
f & w & w
\end{Bmatrix}
$$

$$
\begin{vmatrix}
a & b & cv \\
5 & 4 & 9 \\
f & w & w
\end{vmatrix}
$$

$$
\begin{Vmatrix}
a & b & cv \\
5 & 4 & 9 \\
\hdotsfor{3}
\end{Vmatrix}
$$

$$
\sqrt[\uproot{10}\leftroot{7}n/2]{\frac{1}{\sum\limits_{i=1}^b a_i}}
$$

$$
\boxed{a+b=c}
$$

$$
\overleftarrow{abc}\quad \underrightarrow{xyz}
$$

$$
A\xleftarrow[c]{algo}B
$$

$$
\overset{a}{B}\quad \underset{\circ}{D}
$$	


texto texto $\dfrac{a}{b}$ texto $\dbinom{a}{b}$ texto texto $\binom{x}{y}$
$$
\tfrac{a}{b}\quad \tbinom{a}{b}
$$

$$
\biggl(\sum_{i=1}^n a_i \biggr)
$$

%\DeclareMathOperator{\sen}{sen}
$$
\sen \alpha
$$

%\DeclareMathOperator*{\algo}{algo}
$$
\algo_{x\to 0} x
$$

$$
\sum_{\substack{i=1\\i\neq n/2}}^n a_i
$$

$$
\sum_{\begin{subarray}{l}i=1\\i\neq n/2\end{subarray}}^n a_i
$$

$$
\sideset{_a^b}{_c^d}\sum
$$

$$
\int\!\!\int f(x,y)d(x,y)
$$

$$
\iint f(x,y)d(x,y)\iiint f\iiiint f\idotsint_{R^n} g
$$


	
	
	
	
\chapter{Algo de mate}
\section{primero}
	
$$
\boldsymbol{\alpha} +\alpha
$$

$$
\vec{x}=(x_1,x_2,x_3)
$$

$$
\boldsymbol{x}-\mathbf{x}=(x_1,x_2,x_3)
$$

$$
\pmb{x}x\boldsymbol{x}
$$

\begin{equation*}
a+b+c=de
\end{equation*}

\begin{multline}
a+b+c+d+a+b+c+d+a+b\\+c+d+a+b+c+d+a+b+c+d+a+\\
b+c+d+a+b+c+d+a+b+c+d
\end{multline}

\begin{multline*}
a+b+c+d+a+b+c+d+a+b\\+c+d+a+b+c+d+a+b+c+d+a+\\
b+c+d+a+b+c+d+a+b+c+d
\end{multline*}

\begin{equation}
\begin{split}
a+b+c+d+a+b+c+d=\alpha \\
a+b+c+d+a+b+c+da+b+c+d+a+b+c+d=\beta
\end{split}
\end{equation}

\begin{gather}
a+b+c+d=e \\
3+4=7 \\
1=1 \\
a+b+c+d+a+b+c+a+b+c+d+a+b+c=2
\end{gather}

\begin{gather*}
a+b+c+d=e \\
3+4=7 \\
1=1 \\
a+b+c+d+a+b+c+a+b+c+d+a+b+c=2
\end{gather*}

\begin{align}
(a+b)^2&=(a+b)(a+b) \\
 &=a^2+ab+ba+b^2 \\
 &=a^2+ab+ab+b^2 \\
 &=a^2+2ab+b^2
\end{align}

\begin{align}
a+3&=fr &  j+2&=444 \\
1&=2 & 32&=4+5
\end{align}

\begin{align*}
a+3&=fr &  j+2&=444 \\
1&=2 & 32&=4+5
\end{align*}

\begin{flalign}
a+3&=fr &  j+2&=444 \\
1&=2 & 32&=4+5
\end{flalign}

\begin{flalign*}
a+3&=fr &  j+2&=444 \\
1&=2 & 32&=4+5
\end{flalign*}

\begin{align}
\label{ec1}(a+b)^2&=(a+b)(a+b) \\
&=a^2+ab+ba+b^2 \notag\\
&=a^2+ab+ab+b^2 \notag\\
\label{ec2}&=a^2+2ab+b^2
\end{align}

En las ecuaciones (\ref{ec1}) y (\ref{ec2}) tenemos...

\section{segundo}
\begin{equation}
a+3=5 \tag{ecuación de Maxwell}
\end{equation}

\begin{equation}
a+3=5 \tag*{ecuación de Maxwell}
\end{equation}

\begin{equation}
f(x)=\left\{\begin{aligned}
x^2+1&=2 \\
x&=6
\end{aligned}\right.
\end{equation}

\begin{equation}
f(x)=\left\{\begin{gathered}
x^2+1=2 \\
x=6
\end{gathered}\right.
\end{equation}

\begin{equation}
f(x)=\begin{cases}
x^2 & x\in X \\
x^3 & x\in Y
\end{cases}
\end{equation}

\chapter{continuación de mate}

$$
f(x)=g(x)\mbox{ cuando }x\in X
$$

$$
f(x)=g(x)\text{ cuando }x\in X
$$

$$
Cl_{\mbox{ácido}}+ H_{\mbox{acuoso}}=--- 
$$

$$
Cl_{\text{ácido}}+ H_{\text{acuoso}}=--- 
$$

\begin{align}
x+y&=5643562345 \\
x&=456345634 \\
\intertext{Por lo tanto}
y&=57567354
\end{align}

ecuación (\ref{ec1})  y ecuación \eqref{ec1}







\newpage	
	
\begin{equation}
a+b+c=d
\end{equation}
	
$$
\int_a^b f(x)dx = g(\tilde{x})
$$
	
\begin{eje}
	esto es un ejemplo
\end{eje}

\begin{eje}
	otro ejemplo
\end{eje}
	
	
	
	
	
	
	
	
\begin{center}
	\sffamily\bfseries\color{blue} algo
\end{center}

texto texto	

\begin{impo}
	algo2 algo2
\end{impo}

\cite{*}
\bibliographystyle{acm}
\bibliography{biblio}
	
	
\end{document}