%\documentclass[utf8,spanish,xcolor={svgnames},14pt]{beamer}
\documentclass[utf8,spanish,xcolor={svgnames},14pt,handout]{beamer}

\uselanguage{spanish}
\languagepath{spanish}

%\usetheme{CambridgeUS}
\usetheme{Warsaw}
%\usetheme{Singapore}
%\usecolortheme{albatross}
\usecolortheme{crane}
\useinnertheme{rectangles}
%\useoutertheme{infolines}
\useoutertheme{sidebar}
\usefonttheme[onlymath]{serif}


\title{Mi primera exposición}
\subtitle{En CTCI--UNI}
\author{Juan Pérez}
\institute[UNI]{UNIVERSIDAD NACIONAL DE INGENIERÍA}
\date{13 de febrero del 2019}

\begin{document}
\maketitle

\begin{frame}{Contenido de mi exposición}\transblindsvertical
\tableofcontents
\end{frame}

\section{Mis primeros slides}
\subsection{básicos}
\begin{frame}\transsplithorizontalout
hola
\end{frame}

\begin{frame}{Mi primera diapositiva}{hoy 13 de marzo}\transsplithorizontalout
hola2
\end{frame}

\begin{frame}[t]
hola3
\end{frame}

\begin{frame}[b]
hola4
\end{frame}

\subsection{avanzados}
\begin{frame}[fragile]\transsplitverticalout
\begin{verbatim}
esto es          un programa
\end{verbatim}
\end{frame}

\begin{frame}[plain]
esta es una diapositiva en blanco

\end{frame}


\section{Overlays}
\begin{frame}\transblindsvertical
esta es una diapositiva en blanco00000
\pause

palabra2 \pause
palabra3 \pause
palabra4

\end{frame}

\begin{frame}{Bibliografía}\transsplithorizontalout
\begin{thebibliography}{9}
	\bibitem{Kop04}
	Kopka, Helmut; Daly, Patrick W.
	\newblock {\em Guide to \LaTeX}. 4th ed.
	\newblock Pearson Education, Inc., 2004.
\end{thebibliography}
\end{frame}

\begin{frame}{listas}\transdissolve
\begin{itemize}
	\item hola
	\begin{itemize}
		\item hola
		\item mundo
		\item hoy día
	\end{itemize}
    \item mundo
    \item hoy día
\end{itemize}

\begin{enumerate}[a)]
	\item hola
	\item mundo
	\item hoy día
\end{enumerate}

\begin{description}[MMMMMMMMMMMMMM]
	\item[hola] Seguramente lo habrás aprendido en la escuela primaria: la Tierra describe una
	
	\item[mundo] texto
\end{description}
\end{frame}


\begin{frame}\transsplithorizontalout
En la siguiente ecuación
\begin{equation}
\int f(x)dx =F(x)+K, K\in R
\end{equation}

Seguramente lo habrás aprendido en la \alert{escuela primaria}: la Tierra describe una

\begin{alertenv}
	esto es importante
\end{alertenv}

\end{frame}

\begin{frame}\transdissolve
\begin{block}{}
esto es un bloque
\end{block}

\begin{block}{bloque común}
	esto es un bloque
\end{block}

\begin{exampleblock}{bloque ejemplo}
	esto es un bloque
\end{exampleblock}

\begin{alertblock}{bloque importante}
	esto es un bloque
\end{alertblock}

\end{frame}

\begin{frame}\transsplitverticalout
\begin{theorem}[de Pitágoras]
	esto es un teorema importante
\end{theorem}

\begin{example}[de la clase]
	esto es un ejemplo
\end{example}

\begin{proof}[Prueba]
	esto es una prueba
\end{proof}
\end{frame}

\begin{frame}
\begin{columns}[b]
	\begin{column}{5cm}
		Seguramente lo habrás aprendido en la escuela primaria: la Tierra describe una primaria: la Tierra describe una
	\end{column}

    \begin{column}{5cm}
    	texto texto
    \end{column}
\end{columns}
\end{frame}


\end{document}